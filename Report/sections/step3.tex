%!TEX root = ../Thesis.tex
\section{Step 3}
This step concerns the barrier. The barrier has to stop the cars until all cars are waiting together. Then they should all be released for a round, and wait when they arrive at the barrier next time.

\subsection{Ideas \& Design}
Before and during the implementation, some ideas were discussed. The cars had to stop at the barrier. Then they had to be released, but only until they arrived at the barrier again. Also the cars should stop waiting, when the barrier is turned off. \\
The idea was to make the on and off function quite similar to the on and off function from the Gate-class. The only difference it that the off function had to release the waiting cars.\\ 
The hard problem was, how to do the sync function right. A counter is used to count waiting cars, and a single semaphore is used to make the cars wait. \\
The sync function should keep track of number of cars, and whether the barrier is on or off.

\subsection{Implementation}
The on and off functions were implemented as the on and off functions from the Gate-class almost. The difference is that the off function releases the cars waiting at the barrier. The on function takes the only coconut in the semaphore b.\\
%\fig{barrier} shows the implementation:

As seen, the sync function increases the cars counter, checks on the barrierOn boolean. Nothing happens in the case that the barrier is not on, meaning off. The cars counter will just be decreased again and the car will go on. \\
In the case that the barrier is on, the car will wait for a coconut in the semaphore, b. Until the cars counter reaches 9, the cars will wait. The semaphore b had only a single coconut, which was taken by the on function. When the 9th car comes along, the cars counter will not be less than 9, and it will release a number of coconuts corresponding to the cars counter. This means, that all cars will be released, and the 

\subsection{Extra B}
