%!TEX root = ../Thesis.tex
\section{Appendix}

\subsection{Collision avoidance} \label{sec:collision_avoidance}
The code for the cars to avoid collision can be seen below.
\begin{lstlisting}
for (int i = 0; i < 9; i++) {
  if (i != no) {
    //Checking the position of the other cars 
    //with the new position
    if (position[i].equals(newpos)) {
      free = false;
    }
    //checking the next position of the other 
    //cars with the cars next position
    if (newpos.equals(cd.nextPos(i, position[i]))) {
      if(inAlley){
        if(no<5){
          //Make sure the cars in 
          //the alley goes first
          if (no > i) {
            free = false;
          }
        }else{
          if (no < i) {
            free = false;
          }
        }
   }else{
      //If the cars are outside the alley the car
      // with highest number goes first
      if (no < i) {
        free = false;
      }
    }
  }
}
\end{lstlisting}

\subsection{Alley} \label{sec:alley}

\begin{lstlisting}
class Alley {

  Semaphore u = new Semaphore(4);
  Semaphore d = new Semaphore(1);
  Semaphore b = new Semaphore(1);
  Semaphore a = new Semaphore(1);
  boolean trafficUp;

  public void enter(int no) throws InterruptedException {

	if (no < 5) {
	  if (trafficUp) {
	    if (Integer.parseInt(u.toString()) == 4) {
          try {
            d.P();
          } catch (InterruptedException e) {
            print(no, " has been removed at d.P");
            throw new InterruptedException();
          }
          trafficUp = true;
	    }
        try {
          u.P();
        } catch (InterruptedException e) {
          print(no, " has been removed at u.P");
          u.V();
          throw new InterruptedException();
        }
		
        } else {
          try {
            b.P();
          } catch (InterruptedException e) {
            print(no, " has been removed at b.P");
            throw new InterruptedException();
          }

            if (trafficUp) {
              b.V();
            } else {
              try {
                d.P();
              } catch (InterruptedException e) {
                print(no, " has been removed at d.P");
                b.V();
                throw new InterruptedException();
              }
              b.V();
            }
            try {
              u.P();
            } catch (InterruptedException e) {
              print(no, " has been removed at u.P");
              u.V();
              throw new InterruptedException();
            }
              trafficUp = true;
          }
      } else {
        if (trafficUp) {
          try {
            d.P();
          } catch (InterruptedException e) {
            print(no, " has been removed at d.P");
            throw new InterruptedException();
          }
          try {
            u.P();
          } catch (InterruptedException e) {
            print(no, " has been removed at u.P");
            u.V();
            throw new InterruptedException();
          }
            trafficUp = false;
        } else {
          if (Integer.parseInt(u.toString()) == 4) {
            try {
              d.P();
            } catch (InterruptedException e) {
              print(no, " has been removed at d.P");
              throw new InterruptedException();
            }
            trafficUp = false;
            }
            try {
              u.P();
            } catch (InterruptedException e) {
              print(no, " has been removed at u.P");
              u.V();
              throw new InterruptedException();
          }
        }
      }
		print(no, " ends entering");
	}

    public void leave(int no) 
	    throws InterruptedException {

	  u.V();
	  
      try{ a.P(); }catch(InterruptedException e){
        throw new InterruptedException();
      }
		
      if (Integer.parseInt(u.toString()) == 4) {
        if(Integer.parseInt(d.toString())==0){
		  d.V();
        }
      }
      a.V();
	  print(no, " ends leaving");
	}
}
\end{lstlisting}

\subsection{jSpin}\label{sec:jSpin}
The code from jSpin to run the analysis:
\begin{lstlisting} 
define N		8	/* no. of processes */

short c = 0;
short u = 4;
short d = 1;
short b = 1;
bool trafficUp = false;

/* Declare and instantiate N Counter processes */


inline v(s){
	s++;
}

inline p(s){
	atomic{
		s > 0 -> s--;
	}
}

active [N] proctype Alley () 
{
	c=(c+1)%N;
entry:
	if :: c<4 -> 
		if :: trafficUp == true -> 
			if :: u == 4 -> p(d);
				trafficUp = true;
               :: else -> skip;
			fi;
		p(u);
		:: trafficUp == false -> p(b);
			if :: trafficUp == true ->
				v(b);
			:: trafficUp == false ->
				p(d);
				v(b); 
			fi;
			p(u);
			trafficUp=true;
		:: else -> skip;
		fi;
	:: c>=4 ->
		if :: trafficUp == true -> 
			p(d);
			p(u);
			trafficUp = false;
		:: trafficUp ==false; ->
			if :: u == 4 ->
				p(d);
				trafficUp=false;
			:: else -> skip;
			fi;
			p(u);
		fi;
	:: else -> skip;
	fi;

leave:
	v(u);
	atomic{
	if :: u==4 ->		
		if :: d== 0 -> v(d);
		:: else -> skip;
		fi;
	:: else -> skip;
	fi;
	}
}

/* Invariant check */
active proctype Check ()
{
	(0 <= c && c <= 7) -> assert(true);
	(0 <= u && u <= 4) -> assert(true);
	(0 <= d && d <= 1) -> assert(true);
	(0 <= b && b <= 1) -> assert(true);	
}
\end{lstlisting}

The analysis run with 5 processes:

\begin{lstlisting}
(Spin Version 6.1.0 -- 4 May 2011)
	+ Partial Order Reduction
Full statespace search for:
	never claim         	- (none specified)
	assertion violations	+
	cycle checks       	- (disabled by -DSAFETY)
	invalid end states	+
State-vector 60 byte, depth reached 72, ... errors: 0 ...
 36981340 states, stored
 57403474 states, matched
 94384814 transitions (= stored+matched)
 10018950 atomic steps
hash conflicts:  35003377 (resolved)
Stats on memory usage (in Megabytes):
 3103.598	equivalent memory usage for states 
 (stored*(State-vector + overhead))
 2015.764	actual memory usage for states (compression: 64.95%)
         	state-vector as stored = 29 byte + 28 byte overhead
  256.000	memory used for hash table (-w25)
    0.107	memory used for DFS stack (-m2000)
 2271.064	total actual memory usage
unreached in proctype Alley
	(0 of 91 states)
unreached in proctype Check
	(0 of 9 states)
pan: elapsed time 145 seconds
pan: rate 255784.62 states/second
\end{lstlisting}

The analysis run with 6 process:
\begin{lstlisting}
pan: resizing hashtable to -w29.. pan: out of memory
hint: to reduce memory, recompile with
  -DCOLLAPSE # good, fast compression, or
  -DMA=68   # better/slower compression, or
  -DHC # hash-compaction, approximation
  -DBITSTATE # supertrace, approximation
(Spin Version 6.1.0 -- 4 May 2011)
Warning: Search not completed
	+ Partial Order Reduction
Full statespace search for:
	never claim         	- (none specified)
	assertion violations	+
	cycle checks       	- (disabled by -DSAFETY)
	invalid end states	+
State-vector 68 byte, depth reached 84, ... errors: 0 ...
 2.69e+08 states, stored
4.5920728e+08 states, matched
7.2820728e+08 transitions (= stored+matched)
 79882859 atomic steps
hash conflicts: 2.7405665e+08 (resolved)
Stats on memory usage (in Megabytes):
24627.686	equivalent memory usage for 
states (stored*(State-vector + overhead))
11490.506	actual memory usage for states (compression: 46.66%)
         	state-vector as stored = 17 byte + 28 byte overhead
 4096.000	memory used for hash table (-w29)
    0.107	memory used for DFS stack (-m2000)
    4.537	memory lost to fragmentation
15582.076	total actual memory usage
pan: elapsed time 1.16e+03 seconds
pan: rate 231499.41 states/second
\end{lstlisting}

\subsection{Barrier Semaphore}
\label{sec:barrier}
The on function:
\begin{lstlisting}
public void on() {
  if (!barrierOn) {
    try {
      b.P();
    } catch (InterruptedException e) {
    }
    barrierOn = true;
  }
}
\end{lstlisting}
The off function:
\begin{lstlisting}
public void off() {
  if (barrierOn) {
    for (int i = 0; i < cars + 1; i++) {
      b.V();
    }
    barrierOn = false;
  }
}
\end{lstlisting}

\subsection{Alley Monitor}
\label{sec:alleymonitor}
The enter function:
\begin{lstlisting}
public synchronized void enter(int no) 
    throws InterruptedException {

  if (no < 5) {
    if (trafficUp) {
      cars++;
    } else {
      if (cars == 0) {
        trafficUp = true;
        cars++;
      } else {
        while (!trafficUp && cars > 0) {
          wait();
        }
        trafficUp = true;
        cars++;
      }
    }
  } else {
    if (trafficUp) {
      if (cars == 0) {
        trafficUp = false;
        cars++;
      } else {
        while (trafficUp && cars > 0) {
          wait();
        }
          trafficUp = false;
          cars++;
        }
    } else {
      cars++;
    }
  }
}
\end{lstlisting}
The leave function:
\begin{lstlisting}
public synchronized void leave(int no) {

  cars--;

  if (cars == 0) {
    notifyAll();
  }
}
\end{lstlisting}
